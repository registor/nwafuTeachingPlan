% ========载入tikz功能增强模块========
\usetikzlibrary{matrix,backgrounds}
\usetikzlibrary{positioning}
\usetikzlibrary{shapes.geometric}
\usetikzlibrary{arrows.meta,arrows}
\usetikzlibrary{automata,snakes}
\usetikzlibrary{mindmap}

% ========页眉页脚风格========
\pagestyle{plain}
% ========有时会出现\headheight too small的warning========
\setlength{\headheight}{15pt}

% ========设定目录深度========
\setcounter{tocdepth}{1}%

% ========设定标题(应该修改为用ctexsetup命令设置)========
\setcounter{secnumdepth}{5}
\titleformat{\section}{\centering\sanhao}{\heiti \dybtNR\thesection}{1em}{\heiti}
\titleformat{\subsection}{\xiaosi}{\hskip -2pt \Roman{subsection}}{1em}{\heiti}
\titleformat{\subsubsection}{\xiaosi}{\heiti\zhnumber{\arabic{subsubsection}}、}{1em}{\textbf}
\titleformat{\paragraph}{\xiaosi}{\arabic{paragraph}、}{1em}{}
\titleformat{\subparagraph}{\xiaosi}{(\arabic{subparagraph})}{1em}{}
%调整标题间距
\titlespacing{\subsection}{0pt}{*0}{*0}
\titlespacing{\subsubsection}{0pt}{*0}{*0}
\titlespacing{\section}{0pt}{*0}{*0}
\titlespacing{\paragraph}{0pt}{*0}{*0}
\titlespacing{\subparagraph}{0pt}{*0}{*0}
 
% ========给左旁注黑圆点========
\usepackage{wasysym}
\let\marginparNR\marginpar
\def\marginpar#1{\marginparNR{ \CIRCLE{}   #1  }}

% ========调整列表前后的间距========
\makeatletter
\let\orig@Enumerate=\enumerate
\renewenvironment{enumerate}{\orig@Enumerate}{\vspace{-0.5cm}\endlist}
\let\orig@Itemize=\itemize
\renewenvironment{itemize}{\orig@Itemize}{\vspace{-0.5cm}\endlist}
\makeatother

%% 设定段间距
\setlength{\parskip}{0.3\baselineskip}

%% 设定行距
\linespread{1}

% ========用tikz画边框========  
\def\biankuang{\leavevmode\vbox to0pt{
                \vss\rlap{\hskip 0.8cm
                \tikz \draw(4,0)--(0,0)--(0,-22.1)--(16.7,-22.1)--(16.7,0)--(4,0)--(4,-22.1);           
                }\vskip -22.4cm}}

% ========定义表格列格式符========            
\newcolumntype{M}[1]{>{\sihao \centering\arraybackslash}m{#1}}
\newcolumntype{N}{@{}m{0pt}@{}}

% ========设定floatrow增强包中插图浮动体参数========
\floatsetup[figure]{objectset=centering, margins=centering}%

% ========设定浮动体计数器范围========
\numberwithin{table}{section}
\numberwithin{figure}{section}

% ========设置图表引用标记=======
\labelformat{figure}{\figurename#1}
\labelformat{table}{\tablename#1}
\labelformat{subfigure}{\figurename\thefigure#1}

% ========设置插图路径========
\graphicspath{{figures/}}

%% 自定义命令
% =========================================================
%%%% 设置常用字体字号,与MS Word相对应
%% 一号, 1.4倍行距
\newcommand{\yihao}{\fontsize{26pt}{36pt}\selectfont}
%% 二号, 1.25倍行距
\newcommand{\erhao}{\fontsize{22pt}{28pt}\selectfont}
%% 小二, 单倍行距
\newcommand{\xiaoer}{\fontsize{18pt}{18pt}\selectfont}
%% 三号, 1.5倍行距
\newcommand{\sanhao}{\fontsize{16pt}{24pt}\selectfont}
%% 小三, 1.5倍行距
\newcommand{\xiaosan}{\fontsize{15pt}{22pt}\selectfont}
%% 四号, 1.5倍行距
\newcommand{\sihao}{\fontsize{14pt}{21pt}\selectfont}
%% 半四, 1.5倍行距
\newcommand{\bansi}{\fontsize{13pt}{19.5pt}\selectfont}
%% 小四, 1.5倍行距
\newcommand{\xiaosi}{\fontsize{12pt}{18pt}\selectfont}
%% 大五, 单倍行距
\newcommand{\dawu}{\fontsize{11pt}{11pt}\selectfont}
%% 五号, 单倍行距
\newcommand{\wuhao}{\fontsize{10.5pt}{10.5pt}\selectfont}
% =========================================================
%%%% 专有名称
\newcommand{\kcmc}[1]{\gdef\kcmcNR{#1}}%课程名称
\newcommand{\dymc}[1]{\gdef\dymcNR{#1}}%授课单元名称
\newcommand{\xqfx}[1]{\gdef\xqfxNR{#1}}%学情分析
\newcommand{\jxmb}[1]{\gdef\jxmbNR{#1}}%教学目标
\newcommand{\jxnd}[1]{\gdef\jxndNR{#1}}%教学难点
\newcommand{\jxzd}[1]{\gdef\jxzdNR{#1}}%教学重点
\newcommand{\jxff}[1]{\gdef\jxffNR{#1}}%教学方法
\newcommand{\jxhj}[1]{\gdef\jxhjNR{#1}}%教学后记

\newcommand{\jcmc}[1]{\gdef\jcmcNR{#1}}%教材名称
\newcommand{\cksm}[1]{\gdef\cksmNR{#1}}%参考书名称
\newcommand{\jsxm}[1]{\gdef\jsxmNR{#1}}%教师姓名
\newcommand{\jyzr}[1]{\gdef\jyzrNR{#1}}%教研室主任
\newcommand{\jxdy}[1]{\gdef\jxdyNR{#1}}%教研单元
\newcommand{\jxzc}[1]{\gdef\jxzcNR{#1}}%教师周次
\newcommand{\jcks}[1]{\gdef\jcksNR{#1}}%教材开始页
\newcommand{\jcjs}[1]{\gdef\jcjsNR{#1}}%教材结束页
\newcommand{\ckks}[1]{\gdef\ckksNR{#1}}%参考书开始页
\newcommand{\ckjs}[1]{\gdef\ckjsNR{#1}}%参考书结束页


\newcommand{\skbc}[1]{\gdef\skbcNR{#1}}%授课班次
\newcommand{\skrq}[1]{\gdef\skrqNR{#1}}%授课日期
\newcommand{\dybt}[1]{\gdef\dybtNR{#1}}%授课单元标题

\newcommand{\jxdw}[1]{\gdef\jxdwNR{#1}}%教学单位(院/系)
% =========================================================
%%%% 教案首页表格(参考高星的代码:https://github.com/gnixoag/myjiaoan)
\newcounter{thesectionSY}
\newcommand{\makeshouye}{
        \setcounter{thesectionSY}{\thesection+1}
        \restoregeometry        
        \renewcommand{\headrulewidth}{0pt}
        \pagestyle{fancy}
        \fancyhead{}
        \lhead{} 
        \chead{
                \begin{tabular}{@{\hspace{1.2cm}}M{7cm}@{\hspace{-0.4cm}}M{8cm}N}
                        \parbox{7cm}{\linespread{0.2}
                                \makebox[7cm][s]{\kaishu \sanhao \nwsuaf }\\ 
                                \makebox[7cm][s]{\kaishu \sanhao \jxdwNR }
                        }
                        &  \makebox[6cm][s]{\rule{0pt}{0.9cm}\yihao \heiti 授课计划}\\
                \end{tabular}
        }
        
        \begin{tabular}{M{2.2cm}|M{7cm}|M{5.8cm}N}
          \hline
          \multirow{2}*{
          \rule{0pt}{1.4cm}\parbox[b]{2.cm}{
          \centering 课\hfill 程\hfill 性\hfill
          质\\及\hfill 主\hfill 题}}& \heiti \dybtNR% \thethesectionSY
          &  ~授~课~单~元:\hfill {\heiti \sihao \underline{第 \jxdyNR
            单元}}\hfill~~~&\\[0.6cm] \cline{2-3}%\hfill 签字
          & \heiti \dymcNR &  ~授~课~周~次:\hfill {\heiti \sihao
            \underline{第 \hspace{0.65\ccwd}\jxzcNR \hspace{0.65\ccwd}周}}\hfill~~~&\\[0.3cm]\hline%\hfill 签字

          \multicolumn{3}{l}{
          \begin{minipage}[t]{15cm}%[t][2.0cm]    
            \begin{minipage}[t]{2.5cm}
              \vspace{6pt} \hfill \heiti \sihao 学情分析:
            \end{minipage}\hspace{0.5cm}
            \begin{minipage}[t]{12cm}%[t][1.5cm]
              \vspace{0pt}\sihao \setlength{\baselineskip}{12pt}
              \begin{enumerate}[1、]
                \xqfxNR
              \end{enumerate}
            \end{minipage}
          \end{minipage}
          }\vspace{0.35cm} &\\ %\hline
          
          \multicolumn{3}{l}{
          \begin{minipage}[t]{15cm}%[t][2.0cm]  
            \begin{minipage}[t]{2.5cm}
              \vspace{6pt} \hfill \heiti \sihao 教学目标:
            \end{minipage}\hspace{0.5cm}
            \begin{minipage}[t]{12cm}%[t][1.5cm]
              \vspace{0pt}\sihao \setlength{\baselineskip}{12pt}
              \begin{enumerate}[1、]
                \jxmbNR
              \end{enumerate}
            \end{minipage}
          \end{minipage}
          }\vspace{0.35cm} &\\ %\hline

          \multicolumn{3}{l}{
          \begin{minipage}[t]{15cm}%[t][2.0cm]  
            \begin{minipage}[t]{2.5cm}
              \vspace{6pt} \hfill \heiti \sihao 教学重点:
            \end{minipage}\hspace{0.5cm}
            \begin{minipage}[t]{12cm}%[t][1.5cm]
              \vspace{0pt}\sihao \setlength{\baselineskip}{12pt}
              \begin{enumerate}[1、]
                \jxzdNR
              \end{enumerate}
            \end{minipage}
          \end{minipage}
          }\vspace{0.35cm} &\\ %\hline

          \multicolumn{3}{l}{
          \begin{minipage}[t]{15cm}%[t][2.0cm]    
            \begin{minipage}[t]{2.5cm}
              \vspace{6pt} \hfill \heiti \sihao 教学难点:
            \end{minipage}\hspace{0.5cm}
            \begin{minipage}[t]{12cm}%[t][1.5cm]
              \vspace{0pt}\sihao \setlength{\baselineskip}{12pt}
              \begin{enumerate}[1、]
                \jxndNR
              \end{enumerate}
            \end{minipage}
          \end{minipage}
          }\vspace{0.35cm} &\\ %\hline

          \multicolumn{3}{l}{
          \begin{minipage}[t]{15cm}%[t][2.0cm]    
            \begin{minipage}[t]{2.5cm}
              \vspace{6pt} \hfill \heiti \sihao 教学方法:
            \end{minipage}\hspace{0.5cm}
            \begin{minipage}[t]{12cm}%[t][1.5cm]
              \vspace{0pt}\sihao \setlength{\baselineskip}{12pt}
              \begin{enumerate}[1、]
                \jxffNR
              \end{enumerate}
            \end{minipage}
          \end{minipage}
          }\vspace{0.7cm} &\\ \hline
          
          % \multicolumn{3}{l}{
          % \begin{minipage}[t][5cm][t]{15cm}
          %   \begin{minipage}[t]{2.5cm}
          %     \vspace{5pt} \hfill \sihao 教学重点:
          %   \end{minipage}\hspace{0.5cm}
          %   \begin{minipage}[t]{12cm}
          %     \vspace{0pt} \sihao \setlength{\baselineskip}{12pt}
          %     \begin{enumerate}[1、] \jxzdNR \end{enumerate}
          %     \vspace{7pt}
          %   \end{minipage}
          %   \vspace{5pt}
          %   % \hline
          %   \begin{minipage}[t]{2.5cm}
          %     \vspace{6pt} \hfill \sihao 教学难点:
          %   \end{minipage}\hspace{0.5cm}
          %   \begin{minipage}[t]{12cm}
          %     \vspace{0pt} \sihao \setlength{\baselineskip}{12pt}
          %     \begin{enumerate}[1、] \jxndNR \end{enumerate}
          %     \vspace{0pt}
          %   \end{minipage}
          %   \begin{minipage}[t]{2.5cm}
          %     \vspace{6pt} \hfill \sihao 教学方法:
          %   \end{minipage}\hspace{0.5cm}
          %   \begin{minipage}[t]{12cm}
          %     \vspace{0pt} \sihao \setlength{\baselineskip}{12pt}
          %     % \vspace{6pt}\sihao \jjffNR
          %     \begin{enumerate}[1、] \jjffNR \end{enumerate}
          %   \end{minipage}
                                
          %               \end{minipage}
          %             } &\\  \hline

          \multirow{2}*{  \rule{0pt}{1.4cm}\parbox[b]{2.cm}{
          \centering 教\hfill 材\hfill 和\\参\hfill 考\hfill 书 } } &
                     \multicolumn{2}{c}{\sihao \jcmcNR 第\underline{\jcksNR}页$\sim$第\underline{\jcjsNR}页} &\\[0.6cm] \cline{2-3}
                  &  \multicolumn{2}{c}{\sihao \cksmNR 第\underline{\ckksNR}页$\sim$第\underline{\ckjsNR}页} &\\[0.6cm] \hline

          
          % \multirow{2}*{\rule{0pt}{1.4cm}\parbox[b]{2.cm}{
          % \centering 授\hfill 课\hfill 班\hfill 次\\授\hfill 课\hfill 日\hfill 期 } } & \multicolumn{2}{c}{ \skbcNR } &\\[0.6cm] \cline{2-3}
          %                           &\multicolumn{2}{c}{ \skrqNR } &\\[0.6cm] \hline
                
          \multicolumn{3}{l}{
          \begin{minipage}[t]{2.5cm}%[t][4.0cm]
            \vspace{0pt} \hfill \heiti \sihao 教学后记:
          \end{minipage}\hspace{0.5cm}
          \begin{minipage}[t]{12cm}%[t][4.5cm]
            \vspace{0pt}\sihao \jxhjNR
          \end{minipage}
          }\vspace{0.7cm} &\\ %\hline      
        \end{tabular}
        \newpage
        \newgeometry{textwidth={\textwidth-150pt},top=2cm,bottom=2cm,right=2.5cm,includehead,includefoot,marginparsep=28pt,marginparwidth=85pt}
        
        \reversemarginpar \fancyhead{} \chead{\hspace{0.5cm} \kaishu
          \yihao 教 \hspace{0.5cm} 案 \hspace{0.5cm} 内\hspace{0.5cm}
          容}% \hspace{1cm} 纸
        % \lhead{\boxhack \boxhackb } %边框
        \lhead{ \biankuang}%边框
        \xiaosi

\section{ \dymcNR }
}
% =========================================================
%%%% 重定义强调字体的代码
% 解决默认强调字体是italic,此时中文会用楷体代替,
% 在此设置为加粗,注意需要使用etoolbox宏包
\makeatletter
\let\origemph\emph
\newcommand*\emphfont{\normalfont\bfseries}
\DeclareTextFontCommand\@textemph{\emphfont}
\newcommand\textem[1]{%
  \ifdefstrequal{\f@series}{\bfdefault}
    {\@textemph{\CTEXunderline{#1}}}
    {\@textemph{#1}}%
}
\RenewDocumentCommand\emph{s o m}{%
  \IfBooleanTF{#1}
    {\textem{#3}}
    {\IfNoValueTF{#2}
      {\textem{#3}\index{#3}}
      {\textem{#3}\index{#2}}%
     }%
}
\makeatother 

%%%% 定义提醒字体
\newcommand{\alert}[1]{\textcolor{red}{\textbf{#1}}}

%%%%% 定义引号命令
\newcommand{\qtmark}[1]{``#1''}

%%%% 定义带引号的加粗强调命令
\newcommand{\qtb}[1]{\qtmark{\emph{#1}}}
%%%% 定义带引号的加粗加红强调命令
\newcommand{\qtbr}[1]{\qtmark{\emph{\alert{#1}}}}

%%%% 中文破折号,据说来自清华模板
\newcommand{\pozhehao}{\kern0.3ex\rule[0.8ex]{2em}{0.1ex}\kern0.3ex}

%叉号与对号,需要用到pifont宏包
\newcommand{\goodmark}{\textcolor{green!50!black}{\Pisymbol{pzd}{52}}}
\newcommand{\badmark}{\textcolor{red}{\Pisymbol{pzd}{56}}}

% =========================================================
%%%% 西北农林科技大学各单位名称
\newcommand{\nwsuaf}{西北农林科技大学}
\newcommand{\cie}{信息工程学院}
\newcommand{\ca}{农学院}
\newcommand{\cpp}{植物保护学院}
\newcommand{\ch}{园艺学院}
\newcommand{\cast}{动物科技学院}
\newcommand{\cvm}{动物医学院}
\newcommand{\cf}{林学院}
\newcommand{\claa}{风景园林艺术学院}
\newcommand{\cnre}{资源环境学院}
\newcommand{\cwrae}{水利与建筑工程学院}
\newcommand{\cmee}{机械与电子工程学院}
\newcommand{\cfse}{食品科学与工程学院}
\newcommand{\ce}{葡萄酒学院}
\newcommand{\cls}{生命科学学院}
\newcommand{\cs}{理学院}
\newcommand{\ccp}{化学与药学院}
\newcommand{\cem}{经济管理学院}
\newcommand{\cm}{马克思主义学院}
\newcommand{\dfl}{外语系}
\newcommand{\iec}{创新实验学院}
\newcommand{\ci}{国际学院}
\newcommand{\dpe}{体育部}
\newcommand{\cvae}{成人教育}
\newcommand{\iswc}{水土保持研究所}


%%% Local Variables:
%%% mode: latex
%%% TeX-master:"../main.tex"
%%% End:
