% ========字节结构的宏包========= 
\usepackage{bytefield}

% ========排版键盘组合和菜单的宏包=========
\usepackage[os=win]{menukeys}

% ========排版代码的宏包=========
\usepackage{minted}
% =========解决minted包排版代码的跨页问题=========
\usepackage[linecolor=black, topline=true, bottomline=true,
leftline=false, rightline=false,
backgroundcolor=yellow!20!white]{mdframed}

% ========定义minted宏包专用于C语言代码的命令========
%%%% 灰色
\definecolor{listinggray}{gray}{0.92}
%%%% 定义C语言代码参数
\setminted{fontsize=\small, breaklines=true,
  breakautoindent=false, autogobble}
\newmintinline{c}{fontsize=\normalsize}
\newminted{c}{bgcolor=yellow!20!white, frame=lines}
\newmintinline[cintt]{c}{escapeinside=||, fontsize=\normalsize}
\newminted[ctt]{c}{bgcolor=yellow!20!white, escapeinside=||, frame=lines}
\newmintedfile{c}{linenos=true}

%%%% 定义shell脚本语言代码参数
\newmintinline{shell}{fontsize=\normalsize}
\newminted[shell]{sh}{autogobble,fontsize=\small, frame=lines}

%%%% 定义makefile代码参数
\newmintinline{basemake}{fontsize=\normalsize}
\newmintinline[makefileinline]{basemake}{fontsize=\normalsize,escapeinside=!!}
\newminted{basemake}{mathescape,
  autogobble,fontsize=\small,frame=lines}
\newminted[makefilett]{basemake}{mathescape,
  autogobble,fontsize=\small,frame=lines,escapeinside=!!}
\newmintedfile{basemake}{linenos=true}

% ========定义TiKz宏包用于绘制流程图的样式========
\def\smbwd{2cm}        
\tikzstyle{startstop} = [rectangle, rounded corners, minimum width=\smbwd,
    minimum height=0.5cm,text centered, align=center, draw=black, fill=red!30]
\tikzstyle{io} = [trapezium, trapezium left angle=70, trapezium right angle=110, minimum width=\smbwd,
    minimum height=0.5cm, text centered, align=center, draw=black, fill=blue!30]
\tikzstyle{process} = [rectangle, minimum width=\smbwd,
    minimum height=0.5cm, text centered, align=center, draw=black, fill=orange!10]
\tikzstyle{decision} = [diamond, minimum width=\smbwd,
    minimum height=0.5cm, text centered, align=center, draw=black, fill=green!30]
\tikzstyle{arrow} = [thick,->,>=stealth]

\tikzstyle{rectpro} = [rectangle, minimum width=2.5cm, text centered,
align=center, draw=black]

% =======定义TiKz宏包用于绘制链表数据结构的样式========
\tikzstyle{ptr}  = [draw, -latex']
\tikzstyle{head} = [rectangle, draw, text height=3mm, text width=3mm, text centered, node distance=3cm, inner sep=0pt]
\tikzstyle{data} = [rectangle split, rectangle split parts=2, draw,
text centered, minimum height=3em]

%% 自定义命令
% =========================================================
%%%%%%%%%% 绘制链表数据结构需要的命令%%%%%%%%
\newcommand{\data}{data \nodepart{second} \phantom{null}}

% 用于绘制无标记内存图中一个内存块的命令,其开始地址在底部,结束地址在顶部
% 语法:
% \memsec{结束地址}{开始地址}{以行为单位的高度}{盒子中的文字}
\newcommand*\istempaddr[1]{\IfStrEq{}{#1}{}{0X#1}}
\newcommand{\memsec}[4]{%
  % 定义memsection的高度
  \bytefieldsetup{bitheight=#3\baselineskip}%
  \bitbox[]{10}{%
    \texttt{\istempaddr{#1}}%0X#1}% 结束地址
    \\
    % 留出空白
    \vspace{#3\baselineskip}
    \vspace{-1.5\baselineskip}
    \vspace{-#3pt}
    \texttt{\istempaddr{#2}}%0X#2}% 开始地址
  }%
  \bitbox{16}{#4}% 盒子里的内容
}

% 定义绘制有标记内存的命令(基于bytefield宏包)
% 创建计算机内存映像的自定义命令,
% 起始地址位于低部,结束地址位于顶部
% 语法:
% \memsection[标记符号]{结束地址}{起始地址}{内存单元高度}{内存单元中的内容}
% 其中标记符号为可选项,可以是任何符合LaTeX语法的内容
\newcommand{\memsection}[5][]{
  \bytefieldsetup{bitheight=#4\baselineskip} % 内存单元的高度
  \bitbox[]{16}{%标记符号及地址盒子,参数为总宽度
    \raggedleft%右对齐
    \texttt{\istempaddr{#2}}% 结束地址
    \\%换行
    % 设置结束地址与起始地址之间的空白
    \vspace{#4\baselineskip}
    \vspace{-2\baselineskip}
    \vspace{-#4pt} % 
    \texttt{#1}\hspace{1em}
    \texttt{\istempaddr{#3}}% 起始地址
  }~
  \bitbox{16}{#5} % 内存单元盒子
}

% 绘制内存逻辑结构图时用到的命令
\newcommand\memcontent[2][2cm]{%%' 
  \begin{minipage}{#1}
    \centering
    #2
  \end{minipage}}
% =========================================================
%%%% C语言专有名词
\newcommand{\cb}{\texttt{Code::Blocks}}
\newcommand{\mww}{\texttt{MinGW-w64}}
\newcommand{\vs}{\texttt{Visual Studio}}
\newcommand{\db}{\texttt{DEBUG}}
\newcommand{\dbger}{\texttt{Debugger}}
\newcommand{\mfile}{\qtmark{\texttt{Makefile}}}
\newcommand{\gdb}{\texttt{gdb}调试器}
\newcommand{\cdb}{\texttt{cdb}调试器}
\newcommand{\gdbcmd}{\texttt{(gdb)}}
\newcommand{\bug}{\texttt{BUG}}
\newcommand{\ieee}{\texttt{IEEE} 754标准}
% =========================================================
%%%% 微分标志符号
\DeclareMathOperator\dif{d\!}


%%% Local Variables:
%%% mode: latex
%%% TeX-master:"../main.tex"
%%% End:
