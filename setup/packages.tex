%% 常用宏包
% =========设置页边距宏包=========
\usepackage[top=2cm, bottom=2cm, left=2.cm, right=2.cm,includehead,includefoot]{geometry}

% =========控制项目列表=========
\usepackage{enumerate}

% =========多栏显示=========
\usepackage{multicol}

% =========合并表行=========
\usepackage{multirow}

% =========超链接宏包=========
\usepackage[%
    pdfstartview=FitH,%
    CJKbookmarks=true,%
    bookmarks=true,%
    bookmarksnumbered=true,%
    bookmarksopen=true,%
    colorlinks=true,%
    citecolor=blue,%
    linkcolor=blue,%
    anchorcolor=green,%
    urlcolor=blue%
    ]{hyperref}

% =========控制标题宏包=========
\usepackage{titlesec}    

% =========控制表格样式(三线表)=========
\usepackage{booktabs}

% =========控制目录=========
\usepackage{titletoc}

% =========控制字体大小=========
\usepackage{type1cm}

% =========首行缩进(可用\noindent取消某段缩进)=========
\usepackage{indentfirst}

% =========支持彩色文本、底色、文本框等=========
\usepackage{color,xcolor}

% =========数学符号宏包=========
\usepackage{amsmath}

% =========图形宏包=========
\usepackage{graphicx}

% =========浮动体增强宏包(多个图形并排等)=========
\usepackage{floatrow}

% =========用tabu代替 array=========
\usepackage{tabu} 

% =========设计页眉页脚宏包=========
\usepackage{fancyhdr}

% =========辅助宏包=========
\usepackage{calc,marvosym,ifthen,fancybox,url,layout}

% =========pgf/tikz绘图宏包=========
\usepackage{pgf, tikz}

% =========修改交叉引用的=========
\usepackage{varioref}

% =========附录设置宏包=========
\usepackage[title,titletoc,header]{appendix}

% =========行内列表宏包=========
\usepackage{paralist}

% =========标题设置宏包=========
\usepackage[labelsep=quad]{caption}


% =========提供了ltx2e里面命令、环境的一些补丁=========
\usepackage{etoolbox}

% =========参考文献宏包=========
\usepackage[backend=biber,autolang=hyphen,style=gb7714-2015ay,
doi=false,url=false,isbn=false]{biblatex}

% =========特殊符号宏包=========
\usepackage{pifont}

%%% Local Variables:
%%% mode: latex
%%% TeX-master:"../main.tex"
%%% End:
